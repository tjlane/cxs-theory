%% ****** Start of file apstemplate.tex ****** %
%%
%%
%%   This file is part of the APS files in the REVTeX 4 distribution.
%%   Version 4.1r of REVTeX, August 2010
%%
%%
%%   Copyright (c) 2001, 2009, 2010 The American Physical Society.
%%
%%   See the REVTeX 4 README file for restrictions and more information.
%%
%
% This is a template for producing manuscripts for use with REVTEX 4.0
% Copy this file to another name and then work on that file.
% That way, you always have this original template file to use.
%
% Group addresses by affiliation; use superscriptaddress for long
% author lists, or if there are many overlapping affiliations.
% For Phys. Rev. appearance, change preprint to twocolumn.
% Choose pra, prb, prc, prd, pre, prl, prstab, prstper, or rmp for journal
%  Add 'draft' option to mark overfull boxes with black boxes
%  Add 'showpacs' option to make PACS codes appear
%  Add 'showkeys' option to make keywords appear

\documentclass[aps,prl,preprint,groupedaddress]{revtex4-1}
%\documentclass[aps,prl,preprint,superscriptaddress]{revtex4-1}
%\documentclass[aps,prl,reprint,groupedaddress]{revtex4-1}

% You should use BibTeX and apsrev.bst for references
% Choosing a journal automatically selects the correct APS
% BibTeX style file (bst file), so only uncomment the line
% below if necessary.
%\bibliographystyle{apsrev4-1}

\usepackage{amsmath}
\usepackage{MnSymbol}

\def\*#1{\mathbf{#1}}

\begin{document}

% Use the \preprint command to place your local institutional report
% number in the upper righthand corner of the title page in preprint mode.
% Multiple \preprint commands are allowed.
% Use the 'preprintnumbers' class option to override journal defaults
% to display numbers if necessary
%\preprint{}

%Title of paper
\title{Correlated scattering may directly reveal local structure in amorphous matter}

% repeat the \author .. \affiliation  etc. as needed
% \email, \thanks, \homepage, \altaffiliation all apply to the current
% author. Explanatory text should go in the []'s, actual e-mail
% address or url should go in the {}'s for \email and \homepage.
% Please use the appropriate macro foreach each type of information

% \affiliation command applies to all authors since the last
% \affiliation command. The \affiliation command should follow the
% other information
% \affiliation can be followed by \email, \homepage, \thanks as well.
\author{Thomas J. Lane}
%\email[]{Your e-mail address}
%\homepage[]{Your web page}
%\thanks{}
%\altaffiliation{}
\affiliation{}

%Collaboration name if desired (requires use of superscriptaddress
%option in \documentclass). \noaffiliation is required (may also be
%used with the \author command).
%\collaboration can be followed by \email, \homepage, \thanks as well.
%\collaboration{}
%\noaffiliation

\date{\today}

\begin{abstract}
% insert abstract here
\end{abstract}

% insert suggested PACS numbers in braces on next line
\pacs{}
% insert suggested keywords - APS authors don't need to do this
%\keywords{}

%\maketitle must follow title, authors, abstract, \pacs, and \keywords
\maketitle


\section{Introduction}
 
If we expand the electron density of the object under study in terms of the spherical harmonics $Y_{\ell m}$,
\[
\rho( \*r ) = \sum_{\ell m} g_{\ell m} (r) \> Y_{\ell m}(\Omega)
\]
where $g_{\ell m} (r)$ is a radial function. Then by Theorem 1, the Fourier transform of this function is
\[
\hat{\rho} (\*q) = \sum_{\ell m} \> G_{\ell m} (q) \> Y_{\ell m}(\Omega)
\]
where
\[
G_{\ell m} (q) = \sqrt{ (2 \ell + 1) \pi} \> i^{-\ell} \> q^{- \ell + \frac{1}{2} } \> \int_0^{\infty} g_{\ell m} (s) J_{\ell + 1/2} (2 \pi q s) \> s^{\ell + \frac{1}{2}} \> ds
\]
when the scattering $| \hat{\rho}(\*q) |^2$ thus expanded is substituted into (\ref{M}) we obtain
\[
M_{\lambda \mu}(q) = \int Y_{\lambda \mu} (\Omega) \> \Big|
\sum_{\ell m} G_{\ell m} (q) \> Y_{\ell m}(\Omega)
\Big|^2 \> d\Omega
\]
this can be written more productively by in terms of a Gaunt integral
\begin{align*}
M_{\lambda \mu}(q) &= \sum_{\ell m} \sum_{L M} \> G_{\ell m}(q) \> G_{L M}^* (q) \>
\int Y_{\lambda \mu} (\Omega) \> Y_{\ell m} (\Omega) \> Y_{LM}^* (\Omega) \> d\Omega \\
&= \sum_{\ell m} \sum_{L M} \> (-1)^M \> N_{\ell L \lambda} \> G_{\ell m}(q) \> G_{L M}^* (q) \>
\left( \begin{array}{ccc}
\ell & L & \lambda \\
0 & 0 & 0 \\
\end{array} \right)
\left( \begin{array}{ccc}
\ell & L & \lambda \\
m & -M & \mu \\
\end{array} \right)
\end{align*}
where we have employed the Wigner $3j$-symbols and the coefficient
\[
N_{\ell L \lambda} = \sqrt{ \frac{(2 \ell + 1)(2 L + 1)(2 \lambda + 1)}{4 \pi} }
\]
alternatively, we may wish to employ the Wigner-Eckart theorem, in which case this reads
\[
M_{\lambda \mu} = \sum_{\ell L} \langle L || Y_\lambda || \ell \rangle 
\sum_{m M} \> G_{\ell m}(q) \> G_{L M}^* (q) \>  C^{\ell m}_{\lambda \mu LM}
\]
where $\langle L || Y_\lambda || \ell \rangle$ is the \emph{reduced matrix element} independent of $m$ and $M$, and $C^{\ell m}_{\lambda \mu LM}$ is a Clebsh-Gordon coefficient.

Let $h_{\ell m} = | \langle L || Y_\lambda || \ell \rangle \> C^{\ell m}_{\lambda \mu LM} |^{1/2}  \> G_{\ell m}$, so that we may write
\[
M_{\lambda \mu} = \sum_{\ell L} \sum_{m M} h_{\ell m}(q) \> h_{L M} (q)
\exp \big\{ i \theta_{\ell m L M} \big\}
\]
where $\theta$ is the phase of $G_{\ell m}(q) \> G_{L M}^* (q)$. So long as we can rotate the Kam bases such that we can ignore all but the $\mu = 0$ term (PROOF NEEDED), then because $m + M + \mu = 0$, we can write
\[
M_{\lambda \mu} = \sum_{\ell L} \sum_{m} h_{\ell m}(q) \> h_{L (-m)} (q)
\exp \big\{ i \theta_{\ell m L M} \big\}
\]
Let matrix $H = (h_{\ell k})$, where $k = m + 1, \ k = 1... 2\ell+1$ re-indexes $m$, and $1_n$ be a vector with unity in the first $n$ positions, and zero everywhere else. Then we can write the modulus of $M_{\lambda \mu}$ very compactly in Dirac notation
\[
| M_{\lambda \mu} | = \langle 1_\lambda | H \, \Pi \, H^T | 1_\lambda \rangle 
\]
with $\Pi$ a permutation matrix of the form
\[
\Pi = \left[ \begin{array}{ccc}
 &  &  1\\
 & \udots  & \\
1 &  &  \\
\end{array} \right]
\]
such that the multiplication $A \Pi$ swaps the columns of $A$. Note that the elements of $H$ are functions of $\lambda$ and $q$; their indexing defines the dependence on $\ell, m, L$ and $M$.

Now, the Kam coefficients take the form (dbl chk but I think this is only OK for autocorrelator)
\[
C_\lambda (q_1, q_2) =
\langle 1_\lambda | H_1 \, \Pi \, H_1^T | 1_\lambda \rangle 
\langle 1_\lambda | H_2 \, \Pi \, H_2^T | 1_\lambda \rangle 
\]
for an autocorrelator $q_1 = q_2$,
\[
\sqrt{C_\lambda (q, q)} =
\langle 1_\lambda | H \, \Pi \, H^T  | 1_\lambda \rangle 
\]
the vector $\*C_\lambda$ may be written
\[
\sqrt{ \*C_\lambda (q,q) } = T H \, \Pi \, H^T \*1
\]
where $T$ is a matrix with ones in the lower diagonal and zeros otherwise and $\*1$ is a vector of all ones. Note that if found, the $h_{\ell m}$ coefficients completely specify the real-space electron density up to a phase. Further, that phase is constrained to discrete values $\theta_{\ell m L M} = \{ 0, 1, 2, 3 \}$. Thus, these coefficients contain nearly all the information about the electron density. The problem of inferring them, however, is quite poorly conditioned.

Further, the conditioning of the problem can be improved if we relax our inferred variables to be
\[
x_{\ell L} = \sum_m h_{\ell m} h_{L (-m)}
\]
in this case, we obtain
\[
\sqrt{ \*C_\lambda (q,q) } = T X \*1
\]
where our independent variables are now contained in the $\lambda_{max} \times \lambda_{max}$ matrix $X$ (\emph{vs.}~the $\lambda_{max} \times (2\lambda_{max} + 1)$ matrix $H$. $X$ gives the (rotationally smeared) magnitudes of the symmetries along the diagonal, cross terms off diagonal. Further, because the inference is now linear (instead of quadratic), this could be solved directly using the principles of compressed sensing via \emph{e.g.}~basis pursuit. We expect $X$ to be sparse, because ...

NEED:  rotation trick to eliminate $\mu$ is acceptable

NOTES: We have ignored cross correlators. Need to figure out what information loss there is due to ignoring them, and how we might be able to employ that info.

\section{Kam Expansion}

We assume that we measure the coherent scattering from a system consisting of identical copies of a specific particle or molecule with electron density $\rho(\*{r})$. The effects of inter-particle interference or solvent contribution are ignored until later. Let the scattering from a single molecule with orientation $\omega$ to scattering vector $\*{q}$ be
\begin{equation} \label{molecular-scattering}
M( \omega, \*{q} ) = \left| \int \rho(R_\omega \, \*{r}) \> e^{ i \*{q} \cdot \*{r} } d\*r \right|^2
\end{equation}
Note that $\omega$ might be \textit{e.g.}~a tuple of three Euler angles, and $R_\omega$ is a rotation operator. We further assume that during the experiment, we expose the sample to light for a very short period of time (much shorter than rotational diffusion) and then refresh the sample and repeat. One such exposure will be called a ``shot'', and the observed intensity for shot indexed $n$ can be written
\[
I_n ( \*q ) = \sum_{\omega_i \in \Gamma_n} M( \omega_i, \*{q} ) + \mathrm{interference\ terms}
\]
Here, $\Gamma_n$ is the set of orientations for each molecule $i$ for a specific shot $n$. The orientations are assumed to be distributed uniformly on the three dimensional unit ball, and the cardinality of $\Gamma_n$, which we denote $m_n$ (the number of molecules in shot $n$) is Poisson distributed with some mean parameter $m$ that depends on the concentration and beam spot size.

Kam was the first to demonstrate that the correlation function
\begin{equation}\label{correlation}
C( \*q_1, \*q_2 ) = \big\langle \delta I_n ( \*q_1 ) \> \delta I_n ( \*q_2 ) \big\rangle_n
\end{equation}
where $\delta I_n ( \*q ) = I_n ( \*q ) - S( q )$, contained detailed information about the electron density of the system under study. We can demonstrate this connection between the electron density and correlation function as follows.

The first important thing to realize is what is meant by $\langle \cdot \rangle_n$ -- this is an average over two quantities: first, the specific orientations within each shot, and second, the number of molecules in each shot. Using the definition of $I_n$, we can split the first term in an expansion of (\ref{correlation}) into ``coherent'' and ``incoherent'' terms, respectively
\[
I_n ( \*q_1 ) \> I_n ( \*q_2 ) = \sum_{\omega_i \in \Gamma_n} M( \omega_i, \*{q}_1 ) M( \omega_i, \*{q}_2 ) 
+ \sum_{\omega_i \in \Gamma_n}\sum_{\omega_{j \neq i} \in \Gamma_n} M( \omega_i, \*{q}_1 ) M( \omega_j, \*{q}_2 ) 
\]
and then perform the expectation, using the fact that the orientations are identically distributed and independent
\begin{align*}
%
\big\langle I_n ( \*q_1 ) \> I_n ( \*q_2 ) \big\rangle_n
%
&= \int p(m_n) \int p(\vec{\omega}) 
\Big\{ \sum_{\omega_i \in \Gamma_n} M( \omega_i, \*{q}_1 ) M( \omega_i, \*{q}_2 ) \\
& \ \ \ + \sum_{\omega_i \in \Gamma_n}\sum_{\omega_{j \neq i} \in \Gamma_n} M( \omega_i, \*{q}_1 ) M( \omega_j, \*{q}_2 )  \Big\} \> dm_n \\
%
&= \int p(m_n) 
\Big\{ m_n \int M( \omega, \*{q}_1 ) M( \omega, \*{q}_2 ) \> d \omega \\
& \ \ \ +  (m_n^2 - m_n) \> \int M( \omega, \*{q}_1 ) \> d\omega \int M( \omega, \*{q}_2 ) \> d\omega  \Big\} \> dm_n \\
%
&= \int p(m_n) 
\Big\{ m_n \int M( \omega, \*{q}_1 ) M( \omega, \*{q}_2 ) \> d \omega
+  \frac{(m_n^2 - m_n)}{m^2} \> S(q_1) \> S(q_2)  \Big\} \> dm_n
%
\end{align*}
We can expand the integral
\begin{equation}\label{integral}
\int M( \omega, \*{q}_1 ) M( \omega, \*{q}_2 ) \> d \omega
\end{equation}
in the spherical harmonics
\begin{equation}\label{expansion}
M( \omega, \*{q} ) = \sum_{m m' \ell} M_{\ell m} (q) \> Y_{\ell m'} (\Omega) \> R_{m' m}^{\ell} (\omega)
\end{equation}
where $M_{\ell m}$ is a complex coefficient, $ Y_{\ell m'} $ is a spherical harmonic, $\Omega$ is the angular orientation of $\*q$, and $R_{m' m}^{\ell}$ is a Wigner rotation matrix. 

Substituting this into (\ref{integral}) and using first the orthogonality theorem
\[
\int R_{m' m}^{(\ell)} (\omega) \> R_{M' M}^{(L)} (\omega) \> d\omega = \frac{1}{2 \ell + 1} \delta_{\ell L} \delta_{m M} \delta_{m' M'}
\]
and then the addition theorem
\[
\sum_{m'} Y_{\ell m'} (\Omega_1) \> Y_{\ell m'}^* (\Omega_2) = \frac{2 \ell + 1}{ 4 \pi} P_\ell (\cos \psi)
\]
where $P_\ell$ is a Legendre polynomial, and $\psi$ is the angle subtended by $\Omega_1$ and $\Omega_2$, we obtain
\begin{equation}\label{expanded}
\int M( \omega, \*{q}_1 ) M( \omega, \*{q}_2 ) \> d \omega = \frac{1}{4 \pi} \sum_{\ell} P_\ell (\cos \psi) \left[ \> \sum_{m = -\ell}^{\ell}  \> M_{\ell m} (q_1) \> M_{\ell m}^* (q_2) \> \right]
\end{equation}
Finally, substituting this back into
\begin{equation}\label{kamtheorem}
\mathrm{ [NEED FULL EXPRESSION]}
\end{equation}
This is the central result of Kam, which we will call ``Kam's Expansion''. The term in square brackets is a coefficient of an expansion of the correlation in Legendre polynomials. These coefficients,
\[
C_\ell (q_1, q_2) = \sum_{m = -\ell}^{\ell}  \> M_{\ell m} (q_1) \> M_{\ell m}^* (q_2)
\]
can be obtained from an experimentally measured correlation function by simple projection,
\[
C_\ell (q_1, q_2) = 2 \pi (2 \ell + 1 ) \int_{0}^{\pi} C( q_1, q_2, \psi ) \> P_\ell (\cos \psi ) \> \sin \psi \> d \psi
\]

Note that the $\ell = 0$ term corresponds to information that is isotropic, dependent only on the magnitudes $q_1$ and $q_2$. This information can be readily re-introduced into any modeling scheme by treating the structure factor $S(q)$ -- which contains all of this isotropic information -- as a separate source of information in addition to measured correlation functions. Further, by Friedel's theorem, $M(\omega, \*q) = M(\omega, - \*q)$, so we expect $C( \*q_1, \*q_2 ) = C( q_1, q_2, \psi )$ to be even around $\psi = 0$. Consequentially, only terms even in $\ell$ should contribute to the sum in (\ref{expanded-correlation}).


\section*{Model Comparison}


\begin{equation}\label{M}
M_{\ell m} (q) =  \int Y_{\ell m} (\Omega) M( 0, q, \Omega ) \> d \Omega
\end{equation}
via \textit{e.g.}~a finite point-sampling algorithm. The normalization here makes these coefficients directly relatable to those measured by experiment, 
Equation (\ref{expanded-correlation}) shows that said coefficients can be obtained from experiment by projecting the measured correlation function onto the Legendre polynomials,\footnote{Recall the normalization of Legendre polynomials, $\int_{-1}^{1} P_m (x) P_n(x) dx = \frac{2}{2n + 1} \delta_{mn}$}

where we have written $C( \*q_1, \*q_2 ) = C( q_1, q_2, \psi )$ to emphasize the dependence of the correlation function on the angle between the scattering vectors.

Now the coefficients $C_\ell (q_1, q_2)$ can be matched between a computed model and measured experiment. Alternatively, one could attempt direct inversion, as discussed by Kam, and later Kirian, Saldin, Spence \textit{et.~al.}



 \section{Useful Facts}
 
The following theorem directly demonstrates that the angular component of the spherical harmonics are preserved under Fourier transformation. It employs a specific radial component to form the \textit{solid spherical harmonics},
\[
P_{\ell m}(\*x) = \sqrt{ \frac{4 \pi}{2 \ell +1} } \> r^\ell \> Y_{\ell m} (\theta, \phi)
\]
but is general for any radial form of the spherical harmonics in $n$ dimensions.
 
\textbf{Theorem 1} (Stein and Weiss, Thm 3.10, p.~158).
\textit{Suppose $n \geq 2$, and function $f \in L^2(E_n) \cap  L^1(E_n)$ has form $f(\*x) = f_0(x)P_{\ell m}(\*x)$, where $P_{\ell m}(\*x)$ is a solid spherical harmonic. Then $\hat{f}$ has form $\hat{f}(\*x) = F_0(x)P_{\ell m}(\*x)$ where
 \[
 F_0(r) = 2 \pi i^{-\ell} r^{-[(n+2\ell-2)/2]} \int_0^{\infty} f_0(s) \> J_{(n+2\ell-2) / 2} \> (2 \pi r s) \> s^{(n+2\ell)/2} \> ds
 \]
 and $J$ is a Bessel function of the first kind.}
 
We further employ the well-known fact that a spherical harmonic of degree $\ell$ can be represented via a linear combination of spherical harmonics, also of degree $\ell$, in a rotated basis.

\textbf{Lemma.}
\textit{For any linear combination of spherical harmonics of the form $\sum_{m=-\ell}^\ell a_{\ell m} Y_{\ell m} (\Omega)$, and any rotation operator $R_\omega$, there exists a second linear combination of harmonics $\sum_{m=-\ell}^\ell b_{\ell m} Y_{\ell m} (\Omega)$ of same degree $\ell$ such that
\[
\sum_{m=-\ell}^\ell a_{\ell m} Y_{\ell m} (\Omega) = \sum_{M=-\ell}^{\ell} b_{\ell M} \> Y_{\ell M} ( R_\omega \, \Omega )
\]
whose coefficients obey
$\sum_{m=-\ell}^\ell a_{\ell m}^2 = \sum_{M=-\ell}^\ell b_{\ell M}^2$}.

Proof. It was demonstrated in [REF] that any spherical harmonic can be represented in a rotated basis by spherical harmonics of the same degree 
\[
 Y_{\ell m} (\Omega) = \sum_{M=-\ell}^{\ell} d^{(\ell)}_{mM} \> Y_{\ell M} ( R_\omega \, \Omega )
\]
where the coefficients obey $\sum_{M=-\ell}^{\ell} \big[ d^{(\ell)}_{mM} \big]^2 = 1$.

Given a linear combination of such rotated spherical harmonics
\[
\sum_{m=-\ell}^{\ell} a_{\ell m} Y_{\ell m} (\Omega) = \sum_{m=-\ell}^{\ell} a_{\ell m} \sum_{M=-\ell}^{\ell} d^{(\ell)}_{mM} \> Y_{\ell M} ( R_\omega \, \Omega )
\]
we may rearrange the finite sums
\[
\sum_{m=-\ell}^{\ell} a_{\ell m} Y_{\ell m} (\Omega) = \sum_{M=-\ell}^{\ell} A_\ell \> d^{(\ell)}_{mM} \> Y_{\ell M} ( R_\omega \, \Omega )
\]
where we have written $A \equiv \sum_{m=-\ell}^\ell a_{\ell m}^2$. Since $\sum_{M=-\ell}^{\ell} \big[ d^{(\ell)}_{mM} \big]^2 = 1$, it follows directly that $\sum_{M=-\ell}^{\ell} \big[ A_\ell \> d^{(\ell)}_{mM} \big]^2 = \sum_{m=-\ell}^\ell a_{\ell m}^2$. Identifying $A_{\ell} \> d^{(\ell)}_{mM} = b_{\ell M}$ completes the proof.






% Specify following sections are appendices. Use \appendix* if there
% only one appendix.
%\appendix
%\section{}

% If you have acknowledgments, this puts in the proper section head.
%\begin{acknowledgments}
% put your acknowledgments here.
%\end{acknowledgments}

% Create the reference section using BibTeX:
\bibliography{basename of .bib file}

\end{document}
%
% ****** End of file apstemplate.tex ******

